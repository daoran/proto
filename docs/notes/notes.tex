\documentclass{article}
\usepackage[utf8]{inputenc}

\usepackage{amsmath}
\usepackage{amssymb}
\usepackage{xcolor}
\usepackage[pdftex]{graphicx}
\usepackage{hyperref}
\hypersetup{
  colorlinks,
  linktoc=all
  citecolor=black,
  filecolor=black,
  linkcolor=black,
  urlcolor=black
}


% General maths
\newcommand{\real}{{\rm I\!R}}

% Linear algebra
\renewcommand{\Vec}[1]{\mathbf{#1}}
\newcommand{\Mat}[1]{\mathbf{#1}}
\newcommand{\re}{{\mathbb{R}}}
\newcommand{\Zeros}[2]{\Vec{0}_{#1\times#2}}
\newcommand{\I}{\Vec{I}}
\newcommand{\rot}{\Mat{R}}
\newcommand{\quat}{\Vec{q}}
\newcommand{\jac}{\Mat{J}}
\newcommand{\Skew}[1]{\lfloor #1 \enspace \times \rfloor}
\newcommand{\Argmin}[1]{\underset{#1}{\text{argmin }}}
\newcommand{\Transpose}[1]{{#1^{\top}}}

% Frames
% \newcommand{\T}{\mathbf{T}}
% \renewcommand{\frame}{\mathcal{F}}
\newcommand{\body}{{\text{B}}}
\newcommand{\robot}{{\text{R}}}
\newcommand{\world}{{\text{W}}}
\newcommand{\glob}{{\text{G}}}
% -- Kinematic Notation --
% ---- Paul Furgale Kinematic Notation ----
\newcommand{\KineNotationP}[3]{{{#1}_{#2#3}}}
\newcommand{\KineNotationF}[4]{{{{}_{#4}} {#1}_{#2#3}}}
% ---- UPenn Kinematic Notation ----
%\newcommand{KineNotationP}[3]{{{}^{#2}_{#3} {#1}}}
%\newcommand{KineNotationF}[4]{{{}^{#3}_{#2} {#1}}}
% ---- UTIAS Kinematic Notation ----
%\newcommand{KineNotationP}[3]{{{#1}^{#2}_{#3}}}
%\newcommand{KineNotationF}[4]{{{#1}^{#3#4}_{#2}}}
% -- Translation --
\newcommand{\trans}{{\Vec{t}}}
\newcommand{\Trans}[3]{{\KineNotationF{\trans}{#1}{#2}{#3}}}
% -- Position --
\newcommand{\pos}{{\Vec{p}}}
\newcommand{\Pos}[3]{{\KineNotationF{\pos}{#1}{#2}{#3}}}
% -- Velocity --
\newcommand{\vel}{{\Vec{v}}}
\newcommand{\Vel}[3]{{\KineNotationF{\vel}{#1}{#2}{#3}}}
% -- Angular Velocity --
\newcommand{\angvel}{{\boldsymbol{\omega}}}
\newcommand{\AngVel}[3]{{\KineNotationF{\angvel}{#1}{#2}{#3}}}
% -- Acceleration --
\newcommand{\acc}{{\Vec{a}}}
\newcommand{\Acc}[3]{{\KineNotationF{\acc}{#1}{#2}{#3}}}
% -- Rotation --
\newcommand{\rot}{{\Mat{R}}}
\newcommand{\Rot}[2]{{\KineNotationP{\rot}{#1}{#2}}}
% -- Transforms --
\newcommand{\tf}{{\Mat{T}}}
\newcommand{\Tf}[2]{{\KineNotationP{\tf}{#1}{#2}}}

% Variables
\newcommand{\error}{{\Vec{e}}}
\newcommand{\camRot}{{\Vec{R}}}
\newcommand{\camPos}{{\Vec{C}}}
\newcommand{\landmarkPos}{{\Vec{L}}}
\newcommand{\projFunc}{{\Vec{h}}}
\newcommand{\measurement}{{\Vec{z}}}
\newcommand{\estimate}{{\tilde{\Vec{z}}}}

\begin{document}

% TABLE OF CONTENTS
\tableofcontents
\newpage


% NOTATIONS
\section{Notations}

Frames:
\begin{align}
  \text{body frame: } & \body \\
  \text{robot frame: } & \robot \\
  \text{world frame: } & \world \\
  \text{global frame: } & \glob
\end{align}

Translation:
\begin{align}
  \trans \\
  \Trans{\glob}{\body}{\glob}
\end{align}


Position:
\begin{align}
  \pos \\
  \Pos{\glob}{\body}{\glob}
\end{align}


Velocity:
\begin{align}
  \vel \\
  \Vel{\glob}{\body}{\glob}
\end{align}


Angular Velocity:
\begin{align}
  \angvel \\
  \AngVel{\glob}{\body}{\glob}
\end{align}


Acceleration:
\begin{align}
  \acc \\
  \Acc{\glob}{\body}{\glob}
\end{align}

Rotation:
\begin{align}
  \rot \\
  \Rot{\body}{\glob}
\end{align}


Transforms:
\begin{align}
  \tf \\
  \Tf{\body}{\glob}
\end{align}



% ROTATION MATRIX
\section{Rotation Matrix}

Z-Y-X rotation sequence:

\begin{equation}
  \rot_{zyx} =
  \begin{bmatrix}
    c(\psi) c(\theta)
    & c(\psi) s(\theta) s(\phi) - s(\psi) c(\phi)
    & c(\psi) s(\theta) c(\phi) + s(\psi) s(\phi) \\
    s(\psi) c(\theta)
    & s(\psi) s(\theta) s(\phi) + c(\psi) c(\phi)
    & s(\psi) s(\theta) c(\phi) - c(\psi) s(\phi) \\
    -s(\theta) & c(\theta) s(\phi) & c(\theta) c(\phi)
  \end{bmatrix}
\end{equation}



% QUATERNIONS
\section{Quaternions}

A quaternion, $\Vec{q} \in \real^{4}$, generally has the following form
%
\begin{equation}
  \quat = q_{w} + q_{x} \bold{i} + q_{y} \bold{j} + q_{z} \bold{k},
\end{equation}
%
where $\{ q_{w}, q_{x}, q_{y}, q_{z} \} \in \real$ and $\{ \bold{i}, \bold{j},
\bold{k} \}$ are the imaginary numbers satisfying
%
\begin{equation}
\begin{split}
  &\bold{i}^{2}
  = \bold{j}^{2}
  = \bold{k}^{2}
  = \bold{ijk}
  = -1 \\
  \bold{ij} = -\bold{ji} &= \bold{k}, \enspace
  \bold{jk} = -\bold{kj} = \bold{i}, \enspace
  \bold{ki} = -\bold{ik} = \bold{j}
\end{split}
\end{equation}
%
corresponding to the Hamiltonian convention. The quaternion can be written as a
4 element vector consisting of a \textit{real} (\textit{scalar}) part, $q_{w}$,
and \textit{imaginary} (\textit{vector}) part $\quat_{v}$ as,
%
\begin{equation}
  \quat =
  \begin{bmatrix} q_{w} \\ \quat_{v} \end{bmatrix} =
  \begin{bmatrix} q_{w} \\ q_{x} \\ q_{y} \\ q_{z} \end{bmatrix}
\end{equation}
%
% Another popular quaternion convetion is the JPL convention. The difference is
% how the quaternion is represented and



\subsection{Some Quaternion Properties}


\subsubsection{Sum}

Let $\Vec{p}$ and $\Vec{q}$ be two quaternions, the sum of both quaternions is,
%
\begin{equation}
  \Vec{p} \pm \Vec{q} =
  \begin{bmatrix} p_w \\ \Vec{p}_{v} \end{bmatrix}
  \pm
  \begin{bmatrix} q_w \\ \Vec{q}_{v} \end{bmatrix} =
  \begin{bmatrix} p_w \pm q_w \\ \Vec{p}_{v} \pm \Vec{q}_{v} \end{bmatrix}.
\end{equation}
%
The sum between two quaternions $\Vec{p}$ and $\Vec{q}$ is \bold{commutative}
and \bold{associative}.
%
\begin{equation}
  \Vec{p} + \Vec{q} = \Vec{q} + \Vec{p}
\end{equation}
%
\begin{equation}
  \Vec{p} + (\Vec{q} + \Vec{r}) = (\Vec{p} + \Vec{q}) + \Vec{r}
\end{equation}


\subsubsection{Product}

The quaternion multiplication (or product) of two quaternions $\Vec{p}$ and
$\Vec{q}$, denoted by $\otimes$ is defined as
%
\begin{align}
  \Vec{p} \otimes \Vec{q} 
    &=
    (p_w + p_x \bold{i} + p_y \bold{j} + p_z \bold{k})
    (q_w + q_x \bold{i} + q_y \bold{j} + q_z \bold{k}) \\
    &=
    \begin{matrix}
      &(p_w q_w - p_x q_x - p_y q_y - p_z q_z)& \\
      &(p_w q_x + p_x q_w + p_y q_z - p_z q_y)& \bold{i}\\
      &(p_w q_y - p_y q_w + p_z q_x + p_x q_z)& \bold{j}\\
      &(p_w q_z + p_z q_w - p_x q_y + p_y q_x)& \bold{k}\\
    \end{matrix} \\
    &=
    \begin{bmatrix}
      p_w q_w - p_x q_x - p_y q_y - p_z q_z \\
      p_w q_x + q_x p_w + p_y q_z - p_z q_y \\
      p_w q_y - p_y q_w + p_z q_x + p_x q_z \\
      p_w q_z + p_z q_w - p_x q_y + p_y q_x \\
    \end{bmatrix} \\
    &=
    \begin{bmatrix}
      p_w q_w - \Transpose{\Vec{p}_{v}} \Vec{q}_{v} \\
      p_w \Vec{q}_{v} + q_w \Vec{p}_{v} + \Vec{p}_{v} \times \Vec{q}_{v}
    \end{bmatrix}.
\end{align}
%



% COMPUTER VISION
\section{Computer Vision}

\subsection{Pinhole Camera Model}

\begin{equation}
  K =
  \begin{bmatrix}
    f_x & 0 & c_x \\
    0 & f_y & c_y \\
    0 & 0 & 1
  \end{bmatrix}
\end{equation}

\subsection{Radial Tangential Distortion}

\begin{align}
\begin{split}
  k_{\text{radial}} &= 1 + (k_1 r^2) + (k_2 r^4) \\
  x' &= x \cdot k_{\text{radial}} \\
  y' &= y \cdot k_{\text{radial}} \\
  x'' &= x' + (2 p_1 x y + p_2 * (r^2 + 2 x^2)) \\
  y'' &= y' + (p_1 (r^2 + 2 y^2) + 2 p_2 x y)
\end{split}
\end{align}

\subsection{Equi-distant Distortion}

\begin{align}
\begin{split}
  r &= \sqrt{x^{2} + y^{2}} \\
  \theta &= \arctan{(r)} \\
  \theta_d &= \theta (1 + k_1 \theta^2 + k_2 \theta^4 + k_3 \theta^6 + k_4 \theta^8) \\
  x' &= (\theta_d / r) \cdot x \\
  y' &= (\theta_d / r) \cdot y
\end{split}
\end{align}


\end{document}

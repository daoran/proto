\documentclass{article}
\usepackage[utf8]{inputenc}

\usepackage{cite}
\usepackage{amsmath}
\usepackage{amssymb}
\usepackage[dvipsnames]{xcolor}
\usepackage[pdftex]{graphicx}
\usepackage{hyperref}
\hypersetup{
  colorlinks,
  linktoc=all
  citecolor=black,
  filecolor=black,
  linkcolor=black,
  urlcolor=black
}

% Linear algebra
\renewcommand{\Vec}[1]{{\mathbf{#1}}}
\newcommand{\Mat}[1]{{\mathbf{#1}}}
\newcommand{\real}{{\rm I\!R}}
\newcommand{\Zeros}[2]{{\Vec{0}_{#1\times#2}}}
\newcommand{\Norm}[1]{{\|#1\|}}
\newcommand{\I}{{\Mat{I}}}
\newcommand{\quat}{{\Vec{q}}}
\newcommand{\jac}{{\Mat{J}}}
\newcommand{\Skew}[1]{{\lfloor #1 \enspace \times \rfloor}}
\newcommand{\Argmin}[1]{\underset{#1}{{\text{argmin }}}}
\newcommand{\Transpose}[1]{{{#1^{\top}}}}
\newcommand{\Inv}[1]{{{#1^{-1}}}}

% Frames
\newcommand{\tf}{\mathbf{T}}
\renewcommand{\frame}{\mathcal{F}}
\newcommand{\body}{{\text{B}}}
\newcommand{\cam}{{\text{C}}}
\newcommand{\robot}{{\text{R}}}
\newcommand{\sensor}{{\text{S}}}
\newcommand{\world}{{\text{W}}}
\newcommand{\glob}{{\text{G}}}
\newcommand{\fiducial}{{\text{F}}}
% -- Kinematic Notation --
% ---- Paul Furgale Kinematic Notation ----
\newcommand{\KineNotationTransform}[3]{{{#1}_{#2#3}}}
\newcommand{\KineNotation}[3]{{{{}_{#2}} {#1}_{#2#3}}}
\newcommand{\KineNotationPart}[3]{{{{}_{#2}} {#1}_{#3}}}
\newcommand{\KineNotationBare}[2]{{{{}_{#2}} {#1}}}
% -- Translation --
\newcommand{\trans}{{\Vec{r}}}
\newcommand{\Trans}[2]{{\KineNotation{\trans}{#1}{#2}}}
% -- Position --
\newcommand{\pos}{{\Vec{r}}}
\newcommand{\Pos}[2]{{\KineNotation{\pos}{#1}{#2}}}
% -- Velocity --
\newcommand{\vel}{{\Vec{v}}}
\newcommand{\Vel}[2]{{\KineNotation{\vel}{#1}{#2}}}
% -- Angular Velocity --
\newcommand{\angvel}{{\boldsymbol{\omega}}}
\newcommand{\AngVel}[2]{{\KineNotation{\angvel}{#1}{#2}}}
% -- Acceleration --
\newcommand{\acc}{{\Vec{a}}}
\newcommand{\Acc}[2]{{\KineNotation{\acc}{#1}{#2}}}
% -- Rotation --
\newcommand{\rot}{{\Mat{C}}}
\newcommand{\Rot}[2]{{\KineNotationTransform{\rot}{#1}{#2}}}
% -- Transforms --
\newcommand{\tf}{{\Mat{T}}}
\newcommand{\Tf}[2]{{\KineNotationTransform{\tf}{#1}{#2}}}
% -- Point --
\newcommand{\point}{\Vec{p}}
\newcommand{\Pt}[1]{{\KineNotationPart{\point}{#1}{}}}
\newcommand{\Point}[2]{{\KineNotationBare{\point}{#1}}}

% Variables
\newcommand{\error}{{\Vec{e}}}
\newcommand{\camRot}{{\Rot{\world}{\cam}}}
\newcommand{\camPos}{{\Pos{\world}{\cam}}}
\newcommand{\projFunc}{{\Vec{h}}}
\newcommand{\measurement}{{\Vec{z}}}
\newcommand{\estimate}{{\tilde{\Vec{z}}}}


\begin{document}

% TABLE OF CONTENTS
\tableofcontents
\newpage


% NOTATIONS
\section{Notations}

A large part of robotics is about developing machines that perceives and
interact with the environment. For that robots use sensors to collect and
process data, and knowing what the data describes is of utmost importance.
Imagine obtaining the position of the robot but not knowing what that position
is with respect to. Missing data descriptions such as what a position vector is
expressing, what is it with respect to and more causes many hours of painful
trail and error to extract that information.

In the following section the notation used throughout this document will be
described, and it follows closely of that of Paul Furgale's~\cite{Furgale2014}.
The aim is to mitigate the ambiguity that arises when describing robot poses,
sensor data and more.

A vector expressed in the world frame, $\frame_{\world}$, is written as
$\KineNotationBare{\pos}{\world}$, or more precisely if the vector describes
the position of the camera frame, $\frame_{\cam}$, expressed in
$\frame_{\world}$, the vector can be written as $\Pos{\world}{\cam}}$ with
$\world$ and $\cam$ as start and end points, or for brevity as
$\KineNotationPart{\pos}{\world}{\cam}$. Similarly a transformation of a point from
$\frame_{\cam}$ to $\frame_{\world}$ can be represented by a homogeneous
transform matrix, $\Tf{\world}{\cam}$, where its rotation matrix component is
written as $\Rot{\world}{\cam}$ and the translation component written as
$\Trans{\world}{\cam}$. A rotation matrix that is parametrized by quaternion
$\quat_{\world\cam}$ is written as $\rot\{\quat_{\world\cam}\}$.



\begin{align*}
  &\text{Position:} \enspace & \Pos{\world}{\body} \\
  &\text{Velocity:} \enspace & \Vel{\world}{\body} \\
  &\text{Acceleration:} \enspace & \Acc{\world}{\body} \\
  &\text{Angular velocity:} \enspace & \AngVel{\world}{\body} \\
  &\text{Rotation:} \enspace & \Rot{\world}{\body} \\
  &\text{Transform:} \enspace & \Tf{\world}{\body} \\
  &\text{Point:} \enspace & \Pt{\world}
\end{align*}



% ROTATIONS
\newpage
\section{Rotations}

Z-Y-X rotation sequence:

\begin{equation}
  \rot_{zyx} =
  \begin{bmatrix}
    c(\psi) c(\theta)
    & c(\psi) s(\theta) s(\phi) - s(\psi) c(\phi)
    & c(\psi) s(\theta) c(\phi) + s(\psi) s(\phi) \\
    s(\psi) c(\theta)
    & s(\psi) s(\theta) s(\phi) + c(\psi) c(\phi)
    & s(\psi) s(\theta) c(\phi) - c(\psi) s(\phi) \\
    -s(\theta) & c(\theta) s(\phi) & c(\theta) c(\phi)
  \end{bmatrix}
\end{equation}



% QUATERNIONS
\newpage
\section{Quaternions}

A quaternion, $\Vec{q} \in \real^{4}$, generally has the following form
%
\begin{equation}
  \quat = q_{w} + q_{x} \mathbf{i} + q_{y} \mathbf{j} + q_{z} \mathbf{k},
\end{equation}
%
where $\{ q_{w}, q_{x}, q_{y}, q_{z} \} \in \real$ and $\{ \mathbf{i}, \mathbf{j},
\mathbf{k} \}$ are the imaginary numbers satisfying
%
\begin{equation}
\begin{split}
  &\mathbf{i}^{2}
  = \mathbf{j}^{2}
  = \mathbf{k}^{2}
  = \mathbf{ijk}
  = -1 \\
  \mathbf{ij} = -\mathbf{ji} &= \mathbf{k}, \enspace
  \mathbf{jk} = -\mathbf{kj} = \mathbf{i}, \enspace
  \mathbf{ki} = -\mathbf{ik} = \mathbf{j}
\end{split}
\end{equation}
%
corresponding to the Hamiltonian convention. The quaternion can be written as a
4 element vector consisting of a \textit{real} (\textit{scalar}) part, $q_{w}$,
and \textit{imaginary} (\textit{vector}) part $\quat_{v}$ as,
%
\begin{equation}
  \quat =
  \begin{bmatrix} q_{w} \\ \quat_{v} \end{bmatrix} =
  \begin{bmatrix} q_{w} \\ q_{x} \\ q_{y} \\ q_{z} \end{bmatrix}
\end{equation}
%
There are other quaternion conventions, for example, the JPL convention. A more
detailed discussion between Hamiltonian and JPL quaternion convention is
discussed in \cite{Sola2017}.


\subsection{Main Quaternion Properties}


\subsubsection{Sum}

Let $\Vec{p}$ and $\Vec{q}$ be two quaternions, the sum of both quaternions is,
%
\begin{equation}
  \Vec{p} \pm \Vec{q} =
  \begin{bmatrix} p_w \\ \Vec{p}_{v} \end{bmatrix}
  \pm
  \begin{bmatrix} q_w \\ \Vec{q}_{v} \end{bmatrix} =
  \begin{bmatrix} p_w \pm q_w \\ \Vec{p}_{v} \pm \Vec{q}_{v} \end{bmatrix}.
\end{equation}
%
The sum between two quaternions $\Vec{p}$ and $\Vec{q}$ is \textbf{commutative}
and \textbf{associative}.
%
\begin{equation}
  \Vec{p} + \Vec{q} = \Vec{q} + \Vec{p}
\end{equation}
%
\begin{equation}
  \Vec{p} + (\Vec{q} + \Vec{r}) = (\Vec{p} + \Vec{q}) + \Vec{r}
\end{equation}


\subsubsection{Product}

The quaternion multiplication (or product) of two quaternions $\Vec{p}$ and
$\Vec{q}$, denoted by $\otimes$ is defined as
%
\begin{align}
  \Vec{p} \otimes \Vec{q}
    &=
    (p_w + p_x \mathbf{i} + p_y \mathbf{j} + p_z \mathbf{k})
    (q_w + q_x \mathbf{i} + q_y \mathbf{j} + q_z \mathbf{k}) \\
    &=
    \begin{matrix}
      &(p_w q_w - p_x q_x - p_y q_y - p_z q_z)& \\
      &(p_w q_x + p_x q_w + p_y q_z - p_z q_y)& \mathbf{i}\\
      &(p_w q_y - p_y q_w + p_z q_x + p_x q_z)& \mathbf{j}\\
      &(p_w q_z + p_z q_w - p_x q_y + p_y q_x)& \mathbf{k}\\
    \end{matrix} \\
    &=
    \begin{bmatrix}
      \label{eq:quaternion_product}
      p_w q_w - p_x q_x - p_y q_y - p_z q_z \\
      p_w q_x + q_x p_w + p_y q_z - p_z q_y \\
      p_w q_y - p_y q_w + p_z q_x + p_x q_z \\
      p_w q_z + p_z q_w - p_x q_y + p_y q_x \\
    \end{bmatrix} \\
    &=
    \begin{bmatrix}
      \label{eq:quaternion_product_2}
      p_w q_w - \Transpose{\Vec{p}_{v}} \Vec{q}_{v} \\
      p_w \Vec{q}_{v} + q_w \Vec{p}_{v} + \Vec{p}_{v} \times \Vec{q}_{v}
    \end{bmatrix}.
\end{align}
%
The quaternion product is \textbf{not commutative} in the general
case\footnote{There are exceptions to the general non-commutative rule, where
either $\Vec{p}$ or $\Vec{q}$ is real such that $\Vec{p}_{v} \times \Vec{q}_{v}
= 0$, or when both $\Vec{p}_v$ and $\Vec{q}_v$ are parallel, $\Vec{p}_v ||
\Vec{q}_v$. Only in these cirmcumstances is the quaternion product
commutative.},
%
\begin{equation}
  {\Vec{p} \otimes \Vec{q} \neq \Vec{q} \otimes \Vec{p}} \enspace .
\end{equation}
%
The quaternion product is however \textbf{associative},
%
\begin{equation}
  \Vec{p} \otimes (\Vec{q} \otimes \Vec{r})
  = (\Vec{p} \otimes \Vec{q}) \otimes \Vec{r}
\end{equation}
%
and \textbf{distributive over the sum}
%
\begin{equation}
  \Vec{p} \otimes (\Vec{q} + \Vec{r}) =
  \Vec{p} \otimes \Vec{q} + \Vec{p} \otimes \Vec{r}
  \quad \text{and} \quad
  (\Vec{p} \otimes \Vec{q}) + \Vec{r} =
  \Vec{p} \otimes \Vec{r} + \Vec{q} \otimes \Vec{r}
\end{equation}

The quaternion product can alternatively be expressed in matrix form as
%
\begin{equation}
  \Vec{p} \otimes \Vec{q} = [\Vec{p}]_{L} \Vec{q}
  \quad \text{and} \quad
  \Vec{p} \otimes \Vec{q} = [\Vec{q}]_{R} \Vec{p} \enspace ,
\end{equation}
%
where $[\Vec{p}]_{L}$ and $[\Vec{q}]_{R}$ are the left and right
quaternion-product matrices which are derived from
\eqref{eq:quaternion_product},
%
\begin{equation}
  [\Vec{p}]_{L} =
  \begin{bmatrix}
    p_w & -p_x & -p_y & -p_z \\
    p_x & p_w & -p_z & p_y \\
    p_y & p_z & p_w & -p_x \\
    p_z & -p_y & p_x & p_w
  \end{bmatrix},
  \quad \text{and} \quad
  [\Vec{q}]_{R} =
  \begin{bmatrix}
    q_w & -q_x & -q_y & -q_z \\
    q_x & q_w & q_z & -q_y \\
    q_y & -q_z & q_w & q_x \\
    q_z & q_y & -q_x & q_w
  \end{bmatrix},
\end{equation}
%
or inspecting \eqref{eq:quaternion_product_2} a compact form can be derived as,
%
\begin{equation}
  [\Vec{p}]_{L} =
  \begin{bmatrix}
    0 & -\Transpose{\Vec{p}_{v}} \\
    \Vec{p}_w \I_{3 \times 3} + \Vec{p}_{v} &
    \Vec{p}_w \I_{3 \times 3} -\Skew{\Vec{p}_{v}}
  \end{bmatrix}
\end{equation}
%
and
%
\begin{equation}
  [\Vec{q}]_{R} =
  \begin{bmatrix}
    0 & -\Transpose{\Vec{q}_{v}} \\
    \Vec{q}_w \I_{3 \times 3} + \Vec{q}_{v} &
    \Vec{q}_w \I_{3 \times 3} -\Skew{\Vec{q}_{v}}
  \end{bmatrix},
\end{equation}
%
where $\Skew{\bullet}$ is the skew operator that produces a matrix cross
product matrix, and is defined as
%
\begin{equation}
  \Skew{\Vec{v}} =
  \begin{bmatrix}
    0 & -v_{3} & v_{2} \\
    v_{3} & 0 & -v_{1} \\
    -v_{2} & v_{1} & 0
  \end{bmatrix},
  \quad
  \Vec{v} \in \real^{3}
\end{equation}
%

\subsubsection{Conjugate}

The conjugate operator for quaternion, ${(\bullet)}^{\ast}$, is
defined as
%
\begin{equation}
  \quat^{\ast}
  =
  \begin{bmatrix}
    q_w \\
    - \Vec{q}_v
  \end{bmatrix}
  =
  \begin{bmatrix}
    q_w \\
    - q_x \\
    - q_y \\
    - q_z
  \end{bmatrix}.
\end{equation}
%
This has the properties
%
\begin{equation}
  \quat \otimes \quat^{-1}
  = \quat^{-1} \otimes \quat
  = q_{w}^{2} + q_{x}^{2} + q_{y}^{2} + q_{z}^{2}
  =
  \begin{bmatrix}
    q_{w}^{2} + q_{x}^{2} + q_{y}^{2} + q_{z}^{2} \\
    \Vec{0}
  \end{bmatrix},
\end{equation}
%
and
%
\begin{equation}
  (\Vec{p} \otimes \Vec{q})^{\ast}
  = \Vec{q}^{\ast} \otimes \Vec{p}^{\ast}.
\end{equation}


\subsubsection{Norm}

The norm of a quaternion is defined by
%
\begin{align}
  \Norm{\quat} &= \sqrt{\quat \otimes \quat^{\ast}} \\
    &= \sqrt{\quat^{\ast} \otimes \quat} \\
    &= \sqrt{q_{w}^{2} + q_{x}^{2} + q_{y}^{2} + q_{z}^{2}}
    \enspace \in \real,
\end{align}
%
and has the property
%
\begin{align}
  \Norm{\Vec{p} \otimes \Vec{q}} =
  \Norm{\Vec{q} \otimes \Vec{p}} =
  \Norm{\Vec{p}} \Norm{\Vec{q}}
\end{align}



% COMPUTER VISION
\newpage
\section{Computer Vision}

\subsection{Pinhole Camera Model}

\begin{equation}
  K =
  \begin{bmatrix}
    f_x & 0 & c_x \\
    0 & f_y & c_y \\
    0 & 0 & 1
  \end{bmatrix}
\end{equation}



\subsection{Radial Tangential Distortion}

\begin{align}
\begin{split}
  k_{\text{radial}} &= 1 + (k_1 r^2) + (k_2 r^4) \\
  x' &= x \cdot k_{\text{radial}} \\
  y' &= y \cdot k_{\text{radial}} \\
  x'' &= x' + (2 p_1 x y + p_2 (r^2 + 2 x^2)) \\
  y'' &= y' + (p_1 (r^2 + 2 y^2) + 2 p_2 x y)
\end{split}
\end{align}



\subsection{Equi-distant Distortion}

\begin{align}
\begin{split}
  r &= \sqrt{x^{2} + y^{2}} \\
  \theta &= \arctan{(r)} \\
  \theta_d &= \theta (1 + k_1 \theta^2 + k_2 \theta^4 + k_3 \theta^6 + k_4 \theta^8) \\
  x' &= (\theta_d / r) \cdot x \\
  y' &= (\theta_d / r) \cdot y
\end{split}
\end{align}



\subsection{Bundle Adjustment}

Let $\measurement \in \re^{2}$ be the image measurement and $\projFunc(\cdot)
\in \re^{2}$ be the projection function that produces an image projection
$\estimate$. The reprojection error $e$ is defined as the euclidean distance
between $\measurement$ and $\estimate$.

\begin{equation}
  \error = \measurement - \estimate
\end{equation}

Our aim given image measurement $\measurement$ is to find the image projection
$\estimate$ that minimizes the reprojection $e$. The image projection
$\estimate$ in pixels can be represented in homogeneous coordinates with
$u, v, w$ as
%
\begin{equation}
  \estimate
  = \begin{bmatrix} x \\ y \end{bmatrix}
  = \begin{bmatrix} u / w \\ v / w \end{bmatrix}
\end{equation}
%
% Projection function
\begin{align}
  \label{eq:projection_function}
  \begin{bmatrix} u \\ v \\ w \end{bmatrix}
    = \Mat{P} \begin{bmatrix} \Pt{\world} \\ 1 \end{bmatrix}
      = \Mat{K} \camRot [\Pt{\world} - \camPos]
\end{align}
%
where $u, v, w$ is computed by projecting a landmark position $\Pt{\world}$ in
the world frame to the camera's image plane with projection matrix $\Mat{P}$.
The projection matrix, $\Mat{P}$, can be decomposed into the camera intrinsics
matrix, $\Mat{K}$, the camera rotation, $\camRot$, and camera position,
$\camPos$, expressed in the world frame.
%
The orientation of the camera in \eqref{eq:projection_function} is
represented using a rotation matrix. To reduce the optimization parameters
the rotation matrix can be parameterized by a quaternion by using the following
formula,
%
\begin{equation}
  \camRot = \begin{bmatrix}
    % Line 1
    q_{w}^{2} + q_{x}^{2} - q_{y}^{2} - q_{z}^{2}
    & 2 (q_{x} q_{y} - q_{w} q_{z})
    & 2 (q_{x} q_{z} + q_{w} q_{y}) \\
    % Line 2
    2 (q_{x} q_{y} + q_{w} q_{z})
    & q_{w}^{2} - q_{x}^{2} + q_{y}^{2} - q_{z}^{2}
    & 2 (q_{y} q_{z} - q_{w} q_{x}) \\
    % Line 3
    2 (q_{x} q_{z} - q_{w} q_{y})
    & 2 (q_{y} q_{z} + q_{w} q_{x})
    & q_{w}^{2} - q_{x}^{2} - q_{y}^{2} + q_{z}^{2}
  \end{bmatrix}.
\end{equation}
%
By parameterzing the rotation matrix with a quaternion, the optimization
parameters for the camera's orientation is reduced from 9 to 4.

Our objective is to optimize for the camera rotation $\camRot$, camera
position $\camPos$ and 3D landmark position $\Pt{\world}$ in order to
minimize the cost function,
%
\begin{align}
  &\Argmin{\camRot, \camPos, \Pt{\world}} \Norm{
    \measurement - \projFunc(\camRot, \camPos, \Pt{\world})
  }^{2} \\
  &\Argmin{\camRot, \camPos, \Pt{\world}} \Norm{
    \begin{bmatrix} x \\ y \end{bmatrix} -
    \begin{bmatrix} u / w \\ v / w \end{bmatrix}
    }^{2}.
\end{align}
%
The cost function above assumes only a single measurement, if there are $N$
measurements corresponding to $N$ unique landmarks the cost function can be
rewritten as a maximum likelihood estimation problem as,
%
\begin{equation}
  \Argmin{\camRot, \camPos, \Pt{\world}}
  \sum_{j = 1}^{N}
  \Norm{
    \measurement_{j} - \projFunc(\camRot^{j}, \camPos^{j}, \Point{\world}{j})
  }^{2}
\end{equation}
%
under the assumption that the observed landmark, $\Pt{\world}$, measured in
the image plane, $z$, are corrupted by a \textbf{zero-mean Gaussian noise}.

For the general case of $M$ images taken at different camera poses the cost
function can be further extended to,
%
\begin{equation}
  \min_{\camRot, \camPos, \Pt{\world}} 
  \sum_{i = 1}^{M} \sum_{j = 1}^{N}
  \Norm{
    \measurement_{i, j}
    - \projFunc(\camRot^{i}, \camPos^{i}, \Point{\world}{j})
  }^{2}
\end{equation}
%
The optimization process begins by setting the first image camera pose as world
origin, and subsequent $\camRot_{i}$ and $\camPos_{i}$ will be relative to the
first camera pose.


\subsection*{Jacobians}

The Jacobian for the optimization problem for a \textbf{single measurement} has
the form:
%
\begin{equation}
  \jac = \begin{bmatrix}
    \color{red}
    \dfrac{\partial{\projFunc}}{\partial \camRot}
    \color{cyan}
    \dfrac{\partial{\camRot}}{\partial{\quat}} \quad
    \color{Thistle}
    \dfrac{\partial{\projFunc}}{\partial \camPos} \quad
    \color{blue}
    \dfrac{\partial{\projFunc}}{\partial \Pt{\world}}
  \end{bmatrix} \\
\end{equation}
%
If there are two measurements the Jacobian is stacked with the following
pattern:
%
\begin{equation}
  \jac = \begin{bmatrix}
    \text{Image 1}_{2 \times 7}
      & \Zeros{2}{7}
      & \text{3D Point}_{2 \times 3} \\
    \Zeros{2}{7}
      & \text{Image 2}_{2 \times 7}
      & \text{3D Point}_{2 \times 3}
  \end{bmatrix}
\end{equation}


% DERIVATION FOR dh / dR
\subsubsection*{Derivation for
$\color{red}
\dfrac{\partial{\projFunc}}{\partial{\camRot}}$}

\begin{align}
  \color{red}
  \dfrac{\partial{\projFunc}}{\partial{\camRot}} =
  \begin{bmatrix}
    \dfrac{
      w \dfrac{\partial{u}}{\partial{\camRot}} -
      u \dfrac{\partial{w}}{\partial{\camRot}}
    }{w^{2}} \vspace{1.0em} \\
    \dfrac{
      w \dfrac{\partial{v}}{\partial{\camRot}} -
      v \dfrac{\partial{w}}{\partial{\camRot}}
    }{w^{2}} \vspace{0.5em}
  \end{bmatrix}_{2 \times 9}
\end{align}

\begin{align}
  % Line 1
  \begin{bmatrix} u \\ v \\ w \end{bmatrix}
    &= \Mat{K} \camRot [\Pt{\world} - \camPos]
    \nonumber \\
  % Line 2
  \begin{bmatrix} u \\ v \\ w \end{bmatrix}
  &= \begin{bmatrix}
      f_x & 0 & p_x \\
      0 & f_y & p_y \\
      0 & 0 & 1
  \end{bmatrix}
  \begin{bmatrix}
      R_{11} & R_{12} & R_{13} \\
      R_{21} & R_{22} & R_{23} \\
      R_{31} & R_{32} & R_{33}
  \end{bmatrix}
  [\Pt{\world} - \camPos]
  \nonumber
\end{align}

\begin{align}
  \dfrac{\partial{u}}{\partial{\camRot}} &=
      \begin{bmatrix}
        f_{x} (\Pt{\world} - \camPos) &
        \Zeros{1}{3} &
        p_{x} (\Pt{\world} - \camPos)
      \end{bmatrix}_{1 \times 9} \\
  \dfrac{\partial{v}}{\partial{\camRot}} &=
    \begin{bmatrix}
      \Zeros{1}{3} &
      f_{y} (\Pt{\world} - \camPos) &
      p_{y} (\Pt{\world} - \camPos)
  \end{bmatrix}_{1 \times 9} \\
  \dfrac{\partial{w}}{\partial{\camRot}} &=
    \begin{bmatrix}
      \Zeros{1}{3} &
      \Zeros{1}{3} &
      (\Pt{\world} - \camPos)
    \end{bmatrix}_{1 \times 9}
\end{align}


% DERIVATION FOR dR / dq
\subsubsection*{Derivation for \color{cyan}
$\dfrac{\partial{\camRot}}{\partial{\quat}}$}

\begin{align}
  \color{cyan}
  \dfrac{\partial{\camRot}}{\partial{\quat}} =
  \begin{bmatrix}
    \dfrac{\partial{\camRot_{11}}}{\partial{\quat}} \\
    \dfrac{\partial{\camRot_{12}}}{\partial{\quat}} \\
    \vdots \\
    \dfrac{\partial{\camRot_{33}}}{\partial{\quat}}
  \end{bmatrix}_{9 \times 4}
\end{align}

\begin{equation}
  \camRot = \begin{bmatrix}
    % Line 1
    q_{w}^{2} + q_{x}^{2} - q_{y}^{2} - q_{z}^{2}
    & 2 (q_{x} q_{y} - q_{w} q_{z})
    & 2 (q_{x} q_{z} + q_{w} q_{y}) \\
    % Line 2
    2 (q_{x} q_{y} + q_{w} q_{z})
    & q_{w}^{2} - q_{x}^{2} + q_{y}^{2} - q_{z}^{2}
    & 2 (q_{y} q_{z} - q_{w} q_{x}) \\
    % Line 3
    2 (q_{x} q_{z} - q_{w} q_{y})
    & 2 (q_{y} q_{z} + q_{w} q_{x})
    & q_{w}^{2} - q_{x}^{2} - q_{y}^{2} + q_{z}^{2}
  \end{bmatrix}
  \nonumber
\end{equation}

\begin{align}
  % R11
  \dfrac{\camRot_{11}}{\partial{\quat}} &=
    \begin{bmatrix}
    0 & -4q_{y} & -4q_{z} & 0
    \end{bmatrix} \\
  % R12
  \dfrac{\camRot_{12}}{\partial{\quat}} &=
    \begin{bmatrix}
      2q_{y} & 2q_{x} & -2q_{w} & -2q_{z}
    \end{bmatrix} \\
  % R13
  \dfrac{\camRot_{13}}{\partial{\quat}} &=
    \begin{bmatrix}
      2q_{z} & 2q_{w} & 2q_{x} & 2q_{y}
    \end{bmatrix} \\
  % R21
  \dfrac{\camRot_{21}}{\partial{\quat}} &=
  \begin{bmatrix}
    2q_{y} & 2q_{x} & 2q_{w} & 2q_{z}
  \end{bmatrix} \\
  % R22
  \dfrac{\camRot_{22}}{\partial{\quat}} &=
    \begin{bmatrix}
      4q_{x} & 0 & 4q_{z} & 0
    \end{bmatrix} \\
  % R23
  \dfrac{\camRot_{23}}{\partial{\quat}} &=
    \begin{bmatrix}
      2q_{w} & 2q_{z} & 2q_{y} & 2q_{x}
    \end{bmatrix} \\
  % R31
  \dfrac{\camRot_{31}}{\partial{\quat}} &=
    \begin{bmatrix}
      2q_{z} & -2q_{w} & 2q_{x} & -2q_{y}
    \end{bmatrix} \\
  % R32
  \dfrac{\camRot_{32}}{\partial{\quat}} &=
    \begin{bmatrix}
      -4q_{x} & -4q_{y} & 0 & 0
    \end{bmatrix} \\
  % R33
  \dfrac{\camRot_{33}}{\partial{\quat}} &=
    \begin{bmatrix}
      2q_{w} & 2q_{z} & 2q_{y} & 2q_{x}
    \end{bmatrix}
\end{align}


% DERIVATION FOR dh / dL
\subsubsection*{Derivation for
$\color{blue} \dfrac{\partial{\projFunc}}{\partial{\Pt{\world}}}$}

\begin{align}
  \color{blue}
  \dfrac{\partial{\projFunc}}{\partial{\Pt{\world}}} =
  \begin{bmatrix}
    \dfrac{
      w \dfrac{\partial{u}}{\partial{\Pt{\world}}} -
      u \dfrac{\partial{w}}{\partial{\Pt{\world}}}
    }{w^{2}} \vspace{1.0em} \\
    \dfrac{
      w \dfrac{\partial{v}}{\partial{\Pt{\world}}} -
      v \dfrac{\partial{w}}{\partial{\Pt{\world}}}
    }{w^{2}} \vspace{0.5em}
  \end{bmatrix}_{2 \times 3}
\end{align}

\begin{align}
  % Line 1
  \begin{bmatrix} u \\ v \\ w \end{bmatrix}
    &= \Mat{K} \camRot [\Pt{\world} - \camPos]
    \nonumber \\
  % Line 2
  \begin{bmatrix} u \\ v \\ w \end{bmatrix}
  &= \begin{bmatrix}
      f_x & 0 & p_x \\
      0 & f_y & p_y \\
      0 & 0 & 1
  \end{bmatrix}
  \begin{bmatrix}
      R_{11} & R_{12} & R_{13} \\
      R_{21} & R_{22} & R_{23} \\
      R_{31} & R_{32} & R_{33}
  \end{bmatrix}
  [\Pt{\world} - \camPos]
  \nonumber
\end{align}

\begin{align}
  \dfrac{\partial{u}}{\partial{\Pt{\world}}} &=
    \begin{bmatrix}
      f_{x} \camRot_{11} + c_{x} \camRot_{31} &
      f_{x} \camRot_{12} + c_{x} \camRot_{32} &
      f_{x} \camRot_{13} + c_{x} \camRot_{33}
    \end{bmatrix} \\
  \dfrac{\partial{v}}{\partial{\Pt{\world}}} &=
    \begin{bmatrix}
      f_{y} \camRot_{21} + c_{y} \camRot_{31} &
      f_{y} \camRot_{22} + c_{y} \camRot_{32} &
      f_{y} \camRot_{23} + c_{y} \camRot_{33}
    \end{bmatrix} \\
  \dfrac{\partial{w}}{\partial{\Pt{\world}}} &=
    \begin{bmatrix}
      \camRot_{31} &
      \camRot_{32} &
      \camRot_{33}
    \end{bmatrix}
\end{align}


% DERIVATION FOR dh / dC
\subsubsection*{Derivation for \color{Thistle}
$\dfrac{\partial{h}}{\partial{\camPos}}$}

% Partial derivative of projection w.r.t camera position
\begin{align}
  \color{Thistle}
  \dfrac{\partial{\projFunc}}{\partial{\camPos}} =
  \begin{bmatrix}
    \dfrac{
      w \dfrac{\partial{u}}{\partial{\camPos}} -
      u \dfrac{\partial{w}}{\partial{\camPos}}
    }{w^{2}} \vspace{1.0em} \\
    \dfrac{
      w \dfrac{\partial{v}}{\partial{\camPos}} -
      v \dfrac{\partial{w}}{\partial{\camPos}}
    }{w^{2}} \vspace{0.5em}
  \end{bmatrix}_{2 \times 3}
\end{align}

\begin{align}
  % Line 1
  \begin{bmatrix} u \\ v \\ w \end{bmatrix}
    &= \Mat{K} \camRot [\Pt{\world} - \camPos]
    \nonumber \\
  % Line 2
  \begin{bmatrix} u \\ v \\ w \end{bmatrix}
  &= \begin{bmatrix}
      f_x & 0 & p_x \\
      0 & f_y & p_y \\
      0 & 0 & 1
  \end{bmatrix}
  \begin{bmatrix}
      R_{11} & R_{12} & R_{13} \\
      R_{21} & R_{22} & R_{23} \\
      R_{31} & R_{32} & R_{33}
  \end{bmatrix}
  [\Pt{\world} - \camPos]
  \nonumber
\end{align}

\begin{align}
  \dfrac{\partial{u}}{\partial{\camPos}} &=
    -\begin{bmatrix}
      f_{x} \camRot_{11} + c_{x} \camRot_{31} &
      f_{x} \camRot_{12} + c_{x} \camRot_{32} &
      f_{x} \camRot_{13} + c_{x} \camRot_{33}
    \end{bmatrix} \\
  \dfrac{\partial{v}}{\partial{\camPos}} &=
    -\begin{bmatrix}
      f_{y} \camRot_{21} + c_{y} \camRot_{31} &
      f_{y} \camRot_{22} + c_{y} \camRot_{32} &
      f_{y} \camRot_{23} + c_{y} \camRot_{33}
    \end{bmatrix} \\
  \dfrac{\partial{w}}{\partial{\camPos}} &=
    -\begin{bmatrix}
      \camRot_{31} &
      \camRot_{32} &
      \camRot_{33}
    \end{bmatrix}
\end{align}



% CALIBRATION
\newpage
\section{Calibration}

\begin{align}
  \Argmin{\Tf{\world}{\sensor}, \Tf{\sensor}{\cam}, \Tf{\world}{\fiducial}}
  \Norm{
    \measurement
    - \projFunc(\Tf{\world}{\sensor}, \Tf{\sensor}{\cam}, \Tf{\world}{\fiducial})
  }^{2}
\end{align}

\begin{align}
  \hat{\measurement} &=
    \projFunc(\Tf{\world}{\sensor},
              \Tf{\sensor}{\cam},
              \Tf{\world}{\fiducial}) \\
  \Vec{e} &= \measurement - \hat{\measurement}
\end{align}

\begin{align}
  \hat{\measurement}
  &=
  \begin{bmatrix}
    \KineNotationBare{X}{\cam} / \KineNotationBare{Z}{\cam} \\
    \KineNotationBare{Y}{\cam} / \KineNotationBare{Z}{\cam}
  \end{bmatrix} \\
  \Point{\cam}{\fiducial_{i}}
  &=
  \begin{bmatrix}
    \KineNotationBare{X}{\cam} \\
    \KineNotationBare{Y}{\cam} \\
    \KineNotationBare{Z}{\cam}
  \end{bmatrix} \\
  % -- dh / dp_C
  \dfrac{\partial{\projFunc}}{\partial{\Pt{\cam}}}
  &=
  \begin{bmatrix}
    1 & 0 & \KineNotationBare{X}{\cam} / \KineNotationBare{Z}{\cam}^{2} \\
    0 & 1 & \KineNotationBare{Y}{\cam} / \KineNotationBare{Z}{\cam}^{2}
  \end{bmatrix}
\end{align}

\begin{align}
  \Point{\cam}{\fiducial_{ij}}
  &=
  \Tf{\sensor}{\cam}^{-1}
  \Tf{\world}{\sensor}^{-1}
  \Tf{\world}{\fiducial}
  \enspace \Point{\fiducial}{\fiducial_{ij}}
\end{align}


\subsection{Jacobian w.r.t Sensor Pose, $\Tf{\world}{\sensor}$}

\begin{align}
  \Point{\world}{\fiducial_{ij}}
  &=
    \Tf{\world}{\sensor}
    \enspace \Pt{\sensor}
\end{align}

\begin{align}
  \dfrac{\partial{\projFunc}}{\partial{\Pt{\world}}}
  &=
  \dfrac{\partial{\projFunc}}{\partial{\Pt{\cam}}}
  \dfrac{\partial{\Pt{\cam}}}{\partial{\Pt{\world}}} \\
  % -- dh / dp_C
  \dfrac{\partial{\projFunc}}{\partial{\Pt{\cam}}}
  &=
  \begin{bmatrix}
    1 & 0 & \KineNotationBare{X}{\cam} / \KineNotationBare{Z}{\cam}^{2} \\
    0 & 1 & \KineNotationBare{Y}{\cam} / \KineNotationBare{Z}{\cam}^{2}
  \end{bmatrix} \\
  % -- dp_C / dp_S
  \dfrac{\partial{\Pt{\cam}}}{\partial{\Pt{\world}}}
  &=
  \Rot{\cam}{\world}
\end{align}

\begin{align}
  \dfrac{\partial{\projFunc}}{\partial{\Tf{\world}{\sensor}}}
    &=
  \begin{bmatrix}
    \dfrac{\partial{\projFunc}}{\partial{\Pos{\world}{\sensor}}}
    \quad
    \dfrac{\partial{\projFunc}}{\partial{\Rot{\world}{\sensor}}}
  \end{bmatrix} \\
  \dfrac{\partial{\projFunc}}{\partial{\Pos{\world}{\sensor}}}
    &=
      \dfrac{\partial{\projFunc}}{\partial{\Pt{\cam}}}
      \dfrac{\partial{\Pt{\cam}}}{\partial{\Pt{\sensor}}}
      \dfrac{\partial{\Pt{\sensor}}}{\partial{\Pos{\world}{\sensor}}} \\
  \dfrac{\partial{\projFunc}}{\partial{\Rot{\world}{\sensor}}}
    &=
      \dfrac{\partial{\projFunc}}{\partial{\Pt{\cam}}}
      \dfrac{\partial{\Pt{\cam}}}{\partial{\Pt{\sensor}}}
      \dfrac{\partial{\Pt{\sensor}}}{\partial{\Rot{\world}{\sensor}}} \\
    \dfrac{\partial{\Pt{\sensor}}}{\partial{\Pos{\world}{\sensor}}}
    &=
      \I \\
    \dfrac{\partial{\Pt{\sensor}}}{\partial{\Rot{\world}{\sensor}}}
    &= -\Skew{\Rot{\world}{\sensor} \enspace \Pt{\sensor}}
\end{align}


\subsection{Jacobian w.r.t Sensor-Camera Extrinsics, $\Tf{\sensor}{\cam}$}

\begin{align}
  \Point{\sensor}{\fiducial_{ij}}
  &=
    \Tf{\sensor}{\cam}
    \enspace \Pt{\cam}
\end{align}

\begin{align}
  \dfrac{\partial{\projFunc}}{\partial{\Pt{\sensor}}}
  &=
  \dfrac{\partial{\projFunc}}{\partial{\Pt{\cam}}}
  \dfrac{\partial{\Pt{\cam}}}{\partial{\Pt{\sensor}}} \\
  % -- dh / dp_C
  \dfrac{\partial{\projFunc}}{\partial{\Pt{\cam}}}
  &=
  \begin{bmatrix}
    1 & 0 & \KineNotationBare{X}{\cam} / \KineNotationBare{Z}{\cam}^{2} \\
    0 & 1 & \KineNotationBare{Y}{\cam} / \KineNotationBare{Z}{\cam}^{2}
  \end{bmatrix} \\
  % -- dp_C / dp_S
  \dfrac{\partial{\Pt{\cam}}}{\partial{\Pt{\sensor}}}
  &=
  \Rot{\cam}{\sensor}
\end{align}


\begin{align}
  \dfrac{\partial{\projFunc}}{\partial{\Tf{\sensor}{\cam}}}
    &=
  \begin{bmatrix}
    \dfrac{\partial{\projFunc}}{\partial{\Pos{\sensor}{\cam}}}
    \quad
    \dfrac{\partial{\projFunc}}{\partial{\Rot{\sensor}{\cam}}}
  \end{bmatrix} \\
  \dfrac{\partial{\projFunc}}{\partial{\Pos{\sensor}{\cam}}}
    &=
      \dfrac{\partial{\projFunc}}{\partial{\Pt{\cam}}}
      \dfrac{\partial{\Pt{\cam}}}{\partial{\Pt{\sensor}}}
      \dfrac{\partial{\Pt{\sensor}}}{\partial{\Pos{\sensor}{\cam}}} \\
  \dfrac{\partial{\projFunc}}{\partial{\Rot{\sensor}{\cam}}}
    &=
      \dfrac{\partial{\projFunc}}{\partial{\Pt{\cam}}}
      \dfrac{\partial{\Pt{\cam}}}{\partial{\Pt{\sensor}}}
      \dfrac{\partial{\Pt{\sensor}}}{\partial{\Rot{\sensor}{\cam}}} \\
    \dfrac{\partial{\Pt{\sensor}}}{\partial{\Pos{\sensor}{\cam}}}
    &=
      \I \\
    \dfrac{\partial{\Pt{\sensor}}}{\partial{\Rot{\sensor}{\cam}}}
    &= -\Skew{\Rot{\sensor}{\cam} \enspace \Pt{\sensor}}
\end{align}


\subsection{Jacobian w.r.t Fiducial Pose, $\Tf{\world}{\fiducial}$}

\begin{align}
\begin{split}
  \Point{\world}{\fiducial_{ij}}
  &= \Tf{\world}{\fiducial}
     \enspace
     \Point{\fiducial}{\fiducial_{ij}} \\
  &= \Rot{\world}{\fiducial}
     \enspace
     \Point{\fiducial}{\fiducial_{ij}}
     + \Trans{\world}{\fiducial}
\end{split}
\end{align}

\begin{align}
  \dfrac{\partial{\projFunc}}{\partial{\Pt{\world}}}
  &=
  \dfrac{\partial{\projFunc}}{\partial{\Pt{\cam}}}
  \dfrac{\partial{\Pt{\cam}}}{\partial{\Pt{\world}}} \\
  % -- dh / dp_C
  \dfrac{\partial{\projFunc}}{\partial{\Pt{\cam}}}
  &=
  \begin{bmatrix}
    1 & 0 & \KineNotationBare{X}{\cam} / \KineNotationBare{Z}{\cam}^{2} \\
    0 & 1 & \KineNotationBare{Y}{\cam} / \KineNotationBare{Z}{\cam}^{2}
  \end{bmatrix} \\
  % -- dp_C / dp_W
  \dfrac{\partial{\Pt{\cam}}}{\partial{\Pt{\world}}}
  &=
  \Rot{\cam}{\world}
\end{align}

\begin{align}
  \dfrac{\partial{\projFunc}}{\partial{\delta\Vec{\theta}}}
  &=
  \dfrac{\partial{\projFunc}}{\partial{\Pt{\world}}}
  \dfrac{\partial{\Pt{\world}}}{\partial{\delta\Vec{\theta}}} \\
  % -- dh / dp_W
  \dfrac{\partial{\projFunc}}{\partial{\Pt{\world}}}
  &=
  \begin{bmatrix}
    1 & 0 & \KineNotationBare{X}{\cam} / \KineNotationBare{Z}{\cam}^{2} \\
    0 & 1 & \KineNotationBare{Y}{\cam} / \KineNotationBare{Z}{\cam}^{2}
  \end{bmatrix}
  \Rot{\cam}{\fiducial} \\
  % -- dp_W / ddtheta
  \dfrac{\partial{\Pt{\world}}}{\partial{\delta\Vec{\theta}}}
  &= \I
\end{align}

\begin{align}
  \dfrac{\partial{\projFunc}}{\partial{\Pos{\world}{\fiducial}}}
  &=
  \dfrac{\partial{\projFunc}}{\partial{\Pt{\world}}}
  \dfrac{\partial{\Pt{\world}}}{\partial{\Pos{\world}{\fiducial}}} \\
  % -- dh / dp_W
  \dfrac{\partial{\projFunc}}{\partial{\Pt{\world}}}
  &=
  \begin{bmatrix}
    1 & 0 & \KineNotationBare{X}{\cam} / \KineNotationBare{Z}{\cam}^{2} \\
    0 & 1 & \KineNotationBare{Y}{\cam} / \KineNotationBare{Z}{\cam}^{2}
  \end{bmatrix}
  \Rot{\cam}{\fiducial} \\
  % -- dp_W / dr_WF
  \dfrac{\partial{\Pt{\world}}}{\partial{\Pos{\world}{\fiducial}}}
  &=
  -\Skew{\rot\{\delta \theta\} \Pt{\fiducial}}
\end{align}




% BIBLIOGRAPHY
\newpage
\bibliographystyle{ieeetr}
\bibliography{notes}
\end{document}

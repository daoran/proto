\chapter{Notations}

A large part of robotics is about developing machines that perceives and
interact with the environment. For that robots use sensors to collect and
process data, and knowing what the data describes is of utmost importance.
Imagine obtaining the position of the robot but not knowing what that position
is with respect to. Missing data descriptions such as what a position vector is
expressing, what is it with respect to and more causes many hours of painful
trail and error to extract that information.

In the following section the notation used throughout this document will be
described, and it follows closely of that of Paul Furgale's~\cite{Furgale2014}.
The aim is to mitigate the ambiguity that arises when describing robot poses,
sensor data and more.

A vector expressed in the world frame, $\frame_{\world}$, is written as
$\KineNotationBare{\pos}{\world}$, or more precisely if the vector describes
the position of the camera frame, $\frame_{\cam}$, expressed in
$\frame_{\world}$, the vector can be written as $\Pos{\world}{\cam}}$ with
$\world$ and $\cam$ as start and end points, or for brevity as
$\KineNotationPart{\pos}{\world}{\cam}$. Similarly a transformation of a point
from $\frame_{\cam}$ to $\frame_{\world}$ can be represented by a homogeneous
transform matrix, $\Tf{\world}{\cam}$, where its rotation matrix component is
written as $\Rot{\world}{\cam}$ and the translation component written as
$\Trans{\world}{\cam}$. A rotation matrix that is parametrized by quaternion
$\quat_{\world\cam}$ is written as $\rot\{\quat_{\world\cam}\}$.



\begin{align*}
  &\text{Position:} \enspace & \Pos{\world}{\body} \\
  &\text{Velocity:} \enspace & \Vel{\world}{\body} \\
  &\text{Acceleration:} \enspace & \Acc{\world}{\body} \\
  &\text{Angular velocity:} \enspace & \AngVel{\world}{\body} \\
  &\text{Rotation:} \enspace & \Rot{\world}{\body} \\
  &\text{Transform:} \enspace & \Tf{\world}{\body} \\
  &\text{Point:} \enspace & \Pt{\world}
\end{align*}

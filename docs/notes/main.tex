\documentclass[11pt]{article}
\usepackage{amsmath}
\usepackage{amssymb}
\usepackage{amsfonts}

\newcommand{\Mat}[1]{\mathbf{#1}}
\renewcommand{\Vec}[1]{\mathbf{#1}}
\newcommand{\Skew}[1]{\lfloor{#1}\rfloor_{\times}}

\newcommand{\real}{\mathbb{R}}
\newcommand{\hess}{\mathbf{H}}
\newcommand{\tangentSpace}{}
\newcommand{\transpose}{T}
\newcommand{\Frame}[1]{\mathcal{F}_{#1}}
\newcommand{\state}{\mathbf{x}}
\newcommand{\error}{\mathbf{e}}
\newcommand{\State}[1]{\state_{#1}}

\newcommand{\Tr}[1]{\text{trace}({#1})}

\newcommand{\quat}{\mathbf{q}}
\newcommand{\vel}{\mathbf{v}}
\newcommand{\angvel}{\boldsymbol{\omega}}
\newcommand{\accel}{\mathbf{a}}
\newcommand{\bias}{\mathbf{b}}
\newcommand{\macc}{\tilde\accel_{S}}
\newcommand{\mgyr}{\tilde\angvel_{S}}
\newcommand{\tgyr}{\angvel_{S}}

\newcommand{\Quat}[2]{\quat_{#1#2}}
\newcommand{\Vel}[2]{\vel_{#1#2}}
\newcommand{\Accel}[2]{\accel_{#1#2}}
\newcommand{\Bias}[1]{\bias_{#1}}

\newcommand{\tf}{\mathbf{T}}
\newcommand{\trans}{\mathbf{r}}
\newcommand{\rot}{\mathbf{C}}

\newcommand{\Tf}[2]{\tf_{#1#2}}
\newcommand{\Trans}[2]{\trans_{#1#2}}
\newcommand{\Rot}[2]{\rot_{#1#2}}
\newcommand{\EstTrans}[2]{\hat\trans_{#1#2}}
\newcommand{\EstQuat}[2]{\hat\quat_{#1#2}}

\newcommand{\dTrans}[2]{\dot{\trans}_{#1#2}}
\newcommand{\dQuat}[2]{\dot{\quat}_{#1#2}}
\newcommand{\dVel}[2]{\dot{\vel}_{#1#2}}
\newcommand{\dAccel}[2]{\dot{\accel}_{#1#2}}
\newcommand{\dBias}[1]{\dot{\bias}_{#1}}
\newcommand{\dchi}{\delta\boldsymbol{\chi}}

\newcommand{\noise}{\mathbf{w}}
\newcommand{\err}{\mathbf{e}}
\newcommand{\meas}{\mathbf{z}}
\newcommand{\mmodel}{\mathbf{h}}
\newcommand{\cost}{\mathbf{J}}
\newcommand{\weight}{\mathbf{W}}
\newcommand{\gravity}{\mathbf{g}_{W}}


\begin{document}


\section{Introduction}


\section{Nonlinear Least Squares}

\begin{align}
  \min_{\Vec{x}} \cost(\Vec{x})
    &=
      \dfrac{1}{2}
      \sum_{i}
      \Vec{e}_{i}^{\transpose} \Mat{W} \Vec{e}_{i} \\
    &=
      \dfrac{1}{2} \enspace
      \Vec{e}_{i}^{\transpose}(\Vec{x})
      \Mat{W}
      \Vec{e}_{i}(\Vec{x})
\end{align}

where the error function, $\Vec{e}(\cdot)$, depends on the optimization
parameter, $\Vec{x} \in \real^{n}$. The error function,
$\Vec{e}(\cdot)$, has a form of
\begin{align}
  \Vec{e}_{i} =
    \Vec{z} - \Vec{h}(\Vec{x})
\end{align}
is defined as the difference between the measured value, $\Vec{z}$, and
the estimated value calculated using the measurement function,
$\Vec{h}(\cdot)$.  Since the error function, $\Vec{e}(\Vec{x})$, is
non-linear, it is approximated with the first-order Taylor series,
\begin{align}
  \Vec{e}(\Vec{x})
    \approx
      \Vec{e}(\bar{\Vec{x}}) +
      \Mat{E}(\bar{\Vec{x}}) \Delta\Vec{x}
\end{align}
where
\begin{align}
  \Mat{E}(\bar{\Vec{x}}) =
    \dfrac{\partial\Vec{e}(\Vec{x})}{\partial\Vec{x}}
    \bigg\rvert_{\Vec{x}_{k}}
\enspace \enspace
  \Delta{\Vec{x}} = \Vec{x} - \bar{\Vec{x}}
\end{align}
\begin{align}
  \dfrac{\partial{\cost}}{\partial{\Vec{x}}} =
    \dfrac{\partial{\cost}}{\partial{\Vec{e}}}
    \dfrac{\partial{\Vec{e}}}{\partial{\Vec{x}}}
\end{align}

\begin{align}
  \dfrac{\partial{\cost}}{\partial{\Vec{e}}} &=
    \dfrac{1}{2} \Vec{e}^{\transpose}(\Vec{x}) \Mat{W} \Vec{e}(\Vec{x}) =
    \Vec{e}^{\transpose}(\Vec{x}) \Mat{W} \\
  %
  \dfrac{\partial{\Vec{e}}}{\partial{\Vec{x}}} &=
    \Vec{e}(\bar{\Vec{x}}) +
    \Mat{E}(\bar{\Vec{x}}) \Delta\Vec{x} =
    \Mat{E}(\bar{\Vec{x}})
\end{align}

\begin{align}
  \dfrac{\partial{\cost}}{\partial{\Vec{x}}}
    &=
      (\Vec{e}^{\transpose}(\Vec{x}) \Mat{W}) (\Mat{E}(\bar{\Vec{x}})) \\
    % Line 2
    &=
      (
        \Vec{e}(\bar{\Vec{x}}) + \Mat{E}(\bar{\Vec{x}}) \Delta\Vec{x}
      )^{\transpose} \Mat{W}
      \Mat{E}(\bar{\Vec{x}}) \\
    % Line 3
    &=
      \Vec{e}^{\transpose}(\bar{\Vec{x}}) \Mat{W} \Mat{E}(\bar{\Vec{x}})
      + \Delta\Vec{x}^{\transpose}
        \Mat{E}(\bar{\Vec{x}})^{\transpose} \Mat{W} \Mat{E}(\bar{\Vec{x}})
      = 0 \\
    % Line 4
    \Delta\Vec{x}^{\transpose}
      \Mat{E}(\bar{\Vec{x}})^{\transpose} \Mat{W} \Mat{E}(\bar{\Vec{x}})
    &=
      - \Vec{e}^{\transpose}(\bar{\Vec{x}}) \Mat{W} \Mat{E}(\bar{\Vec{x}}) \\
    % Line 5
    \underbrace{
      \Mat{E}(\bar{\Vec{x}})^{\transpose} \Mat{W} \Mat{E}(\bar{\Vec{x}})
    }_{\Mat{H}}
      \Delta\Vec{x}
    &=
    \underbrace{
      - \Mat{E}(\bar{\Vec{x}})^{\transpose} \Mat{W} \Vec{e}(\bar{\Vec{x}})
    }_{\Vec{b}}
\end{align}

Solve the normal equations $\Mat{H}\Delta\Vec{x} = \Vec{b}$ for
$\Delta\Vec{x}$ using the Cholesky or QR-decompositon. Once
$\Delta\Vec{x}$ is found the best estimate $\bar{\Vec{x}}$ can be
updated via,

\begin{align}
  \bar{\Vec{x}}_{k + 1} = \bar{\Vec{x}}_{k} + \Delta\Vec{x}.
\end{align}



\section{Schur Complement}

Let $\Mat{M}$ be a matrix that consists of block matrices
$\Mat{A}$, $\Mat{B}$, $\Mat{C}$, $\Mat{D}$,

\begin{align}
  \Mat{M} =
  \begin{bmatrix}
    \Mat{A} & \Mat{B} \\
    \Mat{C} & \Mat{D}
  \end{bmatrix}
\end{align}

if $\Mat{A}$ is invertible, the Schur's complement of the block
$\Mat{A}$ of the matrix $\Mat{B}$ is the defined by

\begin{aligned}
  \Mat{M}/\Mat{A} &= \Mat{D} - \Mat{C} \Mat{A}^{-1} \Mat{B} \\
  \Mat{M}/\Mat{D} &= \Mat{A} - \Mat{B} \Mat{D}^{-1} \Mat{C}
\end{aligned}



\end{document}
